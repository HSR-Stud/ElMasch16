\section{Schrittmotor}
\subsection{Merkmale}
    \begin{itemize}
        \item Der Schrittwinkel hängt vom Aufbau der Maschine ab und kann zwischen 0.6° und 15° sein.
        \item moderne Arten der Schrittmotoren
        \item Drehmoment bis zu 5 Nm
        \item Werden im unteren Leistungsbereich eingesetzt
        \begin{itemize}
            \item Reluktanz-SM
            \item Hybrid-SM
            \item Permanentmagnet-SM (werden am häufigsten verwendet, da hoher Wirkungsgrad und gute statischen und dynamischen Eigenschaften)
        \end{itemize}
    \end{itemize}
    \textcolor{green}{Vorteil}:
    \begin{itemize}
        \item kostengünstig
        \item praktisch nicht überlastbar
    \end{itemize}

\subsection{Reluktanz-SM}
    \begin{longtable}{| p{.40\textwidth} | p{.60\textwidth} |}
        \firsthline
        \textbf{Aufbau} \newline
        \tabbild[scale=0.5]{images/AufbauReluktanzSM} &	
        \newline
        Stator-Zähne: $ SZ_1, SZ_2, SZ_3, SZ_4$ \newline
        Stator-Wicklungen: $ SW_1, SW_2, SW_3, SW_4 $ \newline
        Rotor-Zähne: $ RZ_1, RZ_2$
        \\ \hline
        
        \textbf{Wirkungsprinzip} \newline
        \tabbild[scale=0.6]{images/WirkPrinzReluktanzSM.JPG}&
        \newline
        Die magnetische Reluktanzkraft ist immer anziehend \newline
        $\Rightarrow$ versucht den Luftspalt zu verkleinern.
        \\ \hline
        
         \newline
        \tabbild[scale=0.5]{images/FLWirkPrinzReluktanzSM.JPG}&
        \newline
        \begin{enumerate}
	   \item Die Spule $ SW_1 $ ,$ SW_3 $ sind an die Quelle angeschlossen. 
       \item Das magnetische Feld der ersten zwei Spulen wird erzeugt.
       \item Die Reluktanzkraft wirkt auf den Rotor um die Luftspalte zu verringern.
       \item  Das mechanische Moment wird erzeugt.
\end{enumerate}
        %\\ \hline
        
        %TODO Passende Grafik einfügen ca S.24 Vorlesung Aktive Wicklungen beachten.
        %Evt mit \multicolumn formatieren
%            \tabbild[scale=0.35]{images/ReluktanzSM1.JPG}\vline
%            \tabbild[scale=0.35]{images/ReluktanzSM2.JPG}&
%            \tabbild[scale=0.35]{images/ReluktanzSM3.JPG}\vline
%            \tabbild[scale=0.35]{images/ReluktanzSM4.JPG}
%            \\ \hline
%            
%            \tabbild[scale=0.35]{images/ReluktanzSM11.JPG}\vline
%            \tabbild[scale=0.35]{images/ReluktanzSM21.JPG}&
%            \tabbild[scale=0.35]{images/ReluktanzSM3.JPG}\vline
%            \tabbild[scale=0.35]{images/ReluktanzSM41.JPG}
        
        \\ \lasthline
    \end{longtable}

\subsection{Permanentmagnet-SM}
    \begin{longtable}{| p{.35\textwidth} | p{.60\textwidth} |}
        \hline
        \textbf{Aufbau} \newline
        \tabbild[scale=0.5]{images/AufbauPMagnetSM.JPG} &	
        \newline
        Stator-Zähne: $ SZ_1, SZ_2, SZ_3, SZ_4$ \newline
        Stator-Wicklungen: $ SW_1, SW_2, SW_3, SW_4 $ \newline
        Rotor-Zähne: $ RZ_1, RZ_2$ \newline \tabbild[scale=0.5]{images/Magnet}
        \\ \hline
        
        \textbf{Grundgleichungen} & %TODO Format 3-Spalten
        Stator-Zahnzahl: $  Z_s $  \newline
        Rotor-Zahnzahl: $ Z_R $  \newline
        Stator-Winkel: $ \alpha_S=\frac{2\pi}{Z_s}$ \quad [rad]  \newline
        Rotor-Winkel: $ \alpha_R=\frac{2\pi}{Z_R}$ \quad [rad]  \newline
        Vollschritt-Winkel: $ \alpha_0 = \alpha_R - \alpha_S $ \newline
        Der Vollschrittwinkel bezeichnet die Bewegung des Rotors pro Einzel-Steuerimpuls. \newline
        Strangzahl: $ m= \frac{Z_s}{Z_S - Z_R} $\newline
        Schrittzahl: $ N_p = \frac{2\pi}{\alpha_0}  $\newline
        Steuerfrequenz: $ f_s = N_p \cdot n $\quad $ \left(n=\left[\dfrac{1}{s}\right]\right) $\newline
        \\ \hline
        
         \newline
        \tabbild[scale=0.4]{images/IndukdqSM.JPG}&
        \newline
        \textcolor{red}{wahre Induktivität} \newline
        \textcolor{blue}{lineare Annäherung} \newline \newline
        N = Windungszahl Statorstrang \newline
        $ A_z $ = Zahnfläche \newline
        $ \delta_{d} = \delta_{q}$ = Höhe des Luftspalts
        \\ \hline            
        %TODO w_zu ist die breite der Zahnüberlappung in q position??ubung 7 
        \textbf{Statorinduktivität}\newline
		d-Achse Parallel\newline
        \tabbild[scale=0.6]{images/StatordSM}&
        \[ L_d = \frac{\varPsi_{md}}{I_1}
        =2N \frac{\varPhi_{md}}{I_1}
        =2N\frac{B_{\delta d}A_z}{I_1}
        =\mu_0 2N\frac{2NI_1A_z}{2\delta_d I_1} \]
        \[\quad =\mu_0 2N^2\frac{A_z}{\delta_d} 
         = \mu_0 2N^2\frac{L \cdot w_s}{\delta_d} \]
        \\ 
                   
        q-Achse Parallel\newline
        \tabbild[scale=0.6]{images/StatorqSM}&
        \[ L_q = \frac{\varPsi_{mq}}{I_1}
        =2N \frac{\varPhi_{mq}}{I_1}
        =2N\frac{B_{\delta q}A_z}{I_1}
        =\mu_0 2N\frac{2NI_1A_z}{2\delta_q I_1}\]
        \[\quad =\mu_0 2N^2\frac{A_z}{\delta_q} 
        = \mu_0 2N^2\frac{L \cdot w_{zu}}{\delta_q} \]
        
        \\ \hline
    \end{longtable}
    \clearpage
    \pagebreak
    
\subsection{V11-12 S43}
\begin{longtable}{| p{.35\textwidth} | p{.60\textwidth} |}
    \firsthline
	\textbf{Drehmoment Herleitung}
    \newline
    \tabbild[scale=0.6]{images/StatordqSM1}&
    $ \varPhi_m(\gamma_r,i) = L(\gamma_r(t)) \cdot i(t) $\newline
    \[ u_{Statorkreis}(t)=R\cdot i(t) + \frac{\diff\varPhi_m}{\diff t}(t) = R \cdot i(t) + \frac{\diff}{\diff t}\left[L(\gamma_r(t))\cdot i(t) \right]\] 
    \[\qquad = R \cdot i(t) + \frac{\diff L}{\diff  \gamma_r} (\gamma_r)\cdot \frac{\diff  \gamma_r}{\diff t} i(t) +L(\gamma_r)\cdot \frac{\diff i}{\diff t}(t)\]
    \[ p_{Statorkreis}(t)=u(t) \cdot i(t) = R\cdot i^2(t)+\frac{\diff L}{\diff Y_r}(\gamma_r)\cdot \omega_r\cdot i^2(t)+L\cdot i(t)\cdot \frac{\diff i}{\diff t}(t) \]
    \\ \hline
    
    \textbf{Elektrische Leistung des Stators}&
    \[ p(t) = R \cdot i^2(t) + \frac{\diff L}{\diff \gamma_r}\cdot \omega_r \cdot i^2(t) + L \cdot i(t) \cdot\frac{\diff i}{\diff t}(t) \]
    \[=p_{Cu}(t)+\frac{\diff w_m}{\diff t}(t)+\frac{1}{2}\cdot \frac{\diff L}{\diff \gamma_r}(\gamma_r)\cdot \omega_r \cdot i^2(t) \]
    \\ \hline
    
    \textbf{Zeitableitung magnetischer Energie}&
    \[ w_m(t) = \frac{1}{2} \cdot L(\gamma_r) \cdot i^2(t) \Rightarrow \frac{\diff }{\diff t} \left( \frac{1}{2} \cdot L(\gamma_r) \cdot i^2(t) \right) \]
    \[=\frac{1}{2}\cdot \frac{\diff L}{\diff \gamma_r}(\gamma_r)\cdot\omega_r \cdot i^2(t) + L \cdot i(t) \cdot \frac{\diff i}{\diff t}(t) \]
    \\ \hline
    
    \textbf{Rotorleistung}&
    \[ p_\delta(t)= \frac{1}{2}\cdot \frac{\diff L}{\diff \gamma_r}(\gamma_r)\cdot \omega_r \cdot i^2(t) \]
    \\ \hline
    
    \textbf{Motormoment}&
    \[ m_M(t)=\frac{p_\delta}{\omega_r}(t)=\frac{1}{2}\cdot\frac{\diff L}{\diff \gamma_r}(\gamma_r)\cdot i^2(t) \]
    \\ \hline
    
    \textbf{Motormoment (lineare Annäherung)}&
    \[ M_M = \frac{1}{2}\frac{L_d - L_q}{\alpha_0}I_1^2 \]
    \\ \hline
    
    \newline
    \tabbild[scale=0.7]{images/IndukdqSMY}&
    \newline
    \tabbild[scale=0.7]{images/MomentdqSMY}
    \\ \hline
    
    \textbf{Betriebsverhalten}\newline
    \[ J_g \frac{\diff \omega_r}{\diff t}= J_g \frac{\omega_s - \omega_1}{T_s} = M_M - M_L \] \newline
    $ \omega_s = 2\pi\frac{n_s}{60}=2\pi\frac{f_s}{N_p}=\alpha_0 f_s $&
    $ J_g $ = Motorbezogenes Trägheitsmoment \newline
    $ M_M $ = Motorbezogenes Drehmoment \newline
    $ M_L $ = Lastmoment \newline
    $ \omega_s $ = Kreisgeschwindigkeit Statorfeld \newline
    $ \omega_1 $ = Kreisgeschwindigkeit des Rotors \newline
    $\Rightarrow  \omega_1 = 0 \Rightarrow$ Motor im Stillstand \newline
    \\ \hline
    
    \textbf{Lastmoment} \newline
    \[ M_L(f_s,\omega_1) = M_M -J_g\frac{\omega_s}{T_s}+J_g\frac{\omega_1}{T_s}\]
    \[=M_m -J_g\alpha_0f_s^2+J_g\omega_1f_s \]
    \[ f_{AG}=\sqrt{\frac{M_M}{J_R \cdot \alpha_0}} \]&
    \newline
    $ F_s = \frac{1}{T_s} $ = Schaltfrequenz\newline
    $ \omega_1 $ = Anfangsgeschwindigkeit \newline
    \\ \hline
    
    \textbf{Anlaufkennlinie}\newline
    \tabbild[scale=0.4]{images/AnlaufkennlinieSM.JPG}&
    $ f_{AG} $ = Anlaufgrenzfrequenz \newline
    $ f_{BG} $ =  Betriebsgrenzfrequenz
    \\ \hline
    
\end{longtable}   

\clearpage
\pagebreak
\section{Schrittmotor}
    \subsection{Merkmale}
        \begin{itemize}
            \item Der Schrittwinkel hängt vom Aufbau der Maschine ab und kann zwischen 0.6° und 15° sein.
            \item moderne Arten der Schrittmotoren
            \begin{itemize}
                \item Reluktanz-SM
                \item Hybrid-SM
                \item Permanentmagnet-SM
            \end{itemize}
        \end{itemize}
        \textcolor{green}{Vorteil}:
        \begin{itemize}
            \item hoher Wirkungsgrad
            \item gute statische und dynamische EIgenschaften
        \end{itemize}
    
    \subsection{Reluktanz-SM}
        \begin{longtable}{| p{.40\textwidth} | p{.60\textwidth} |}
            \firsthline
            \textbf{Aufbau} \newline
            \tabbild[scale=0.5]{images/AufbauReluktanzSM.JPG} &	
            \newline
            Stator-Zähne: $ SZ_1, SZ_2, SZ_3, SZ_4$ \newline
            Stator-Wicklung: $ SW_1, SW_2, SW_3, SW_4 $ \newline
            Rotor-Zähne: $ RZ_1, RZ_2$
            \\ \hline
            
            \textbf{Wirkungsprinzip} \newline
            \tabbild[scale=0.45]{images/WirkPrinzReluktanzSM.JPG}&
            \newline
            Die magentische Reluktanzkraft ist immer anziehend \newline
            $\Rightarrow$ veruscht den Luftspalt zu verkleinern.
            \\ \hline
            
             \newline
            \tabbild[scale=0.5]{images/FLWirkPrinzReluktanzSM.JPG}&
            \newline
            Die Spule $ SW_1 $ ,$ SW_3 $ sind an die Quelle angeschlossen. \newline
            Das magnetische Feld der ersten zwei Spulen wird erzeugt. \newline
            Die Reluktanzkraft wirkt auf den Rotor um die Lluftspalte zu verringern. \newline
            Das mechanische Moment wird erzeugt.
            \\ \hline
            
            %TODO Passende Grafik einfügen ca S.24 Vorlesung Aktive Wicklungen beachten.
            \tabbild[scale=0.35]{images/ReluktanzSM1.JPG}\vline
            \tabbild[scale=0.35]{images/ReluktanzSM2.JPG}&
            \tabbild[scale=0.35]{images/ReluktanzSM3.JPG}\vline
            \tabbild[scale=0.35]{images/ReluktanzSM4.JPG}
            \\ \hline
            
            \tabbild[scale=0.35]{images/ReluktanzSM11.JPG}\vline
            \tabbild[scale=0.35]{images/ReluktanzSM21.JPG}&
            \tabbild[scale=0.35]{images/ReluktanzSM3.JPG}\vline
            \tabbild[scale=0.35]{images/ReluktanzSM41.JPG}
            
            \\ \lasthline
        \end{longtable}
    
    \subsection{Permanentmagnet-SM}
        \begin{longtable}{| p{.40\textwidth} | p{.60\textwidth} |}
            \firsthline
            \textbf{Aufbau} \newline
            \tabbild[scale=0.5]{images/AufbauPMagnetSM.JPG} &	
            \newline
            Stator-Zähne: $ SZ_1, SZ_2, SZ_3, SZ_4$ \newline
            Stator-Wicklung: $ SW_1, SW_2, SW_3, SW_4 $ \newline
            Rotor-Zähne: $ RZ_1, RZ_2$ \newline \newline
            \textbf{Grundgleichnung} \newline
            Stator-Zahnzahl: $  MZ_s $ = 4 \newline
            Rotor-Zahnzahl: $ Z_R $ = 2 \newline
            Stator-Winkel: $ \alpha_S=\frac{2\pi}{Z_s}=90\textdegree $ \newline
            Rotor-Winkel: $ \alpha_R=\frac{2\pi}{Z_R}=180\textdegree $ \newline
            Vollschritt-Winkel: $ \alpha_0 = \alpha_R - \alpha_S $
            \\ \hline
            
            &
            N = Windungszahl Statorstrang \newline
            $ A_z $ = Zahnfläche \newline
            $ \delta_{d/q} $ = Höhe des Luftspalts
            \\ \hline            
            
            \textbf{Drehmoment} \quad d-Achse Parallel\newline
            \tabbild[scale=0.6]{images/StatordSM}&
            \[ L_d = 2N \frac{\varPhi_{md}}{I_1}
            =2N\frac{B_{\delta d}A_z}{I_1}
            =\mu_0 2N\frac{2NI_1A_z}{2\delta_d I_1}
            =\mu_0 2N^2\frac{A_z}{\delta_d} \]
            \\ \hline
            
            \textbf{Drehmoment} \quad q-Achse Parallel\newline
            \tabbild[scale=0.6]{images/StatorqSM}&
            \[ L_d = 2N \frac{\varPhi_{mq}}{I_1}
            =2N\frac{B_{\delta q}A_z}{I_1}
            =\mu_0 2N\frac{2NI_1A_z}{2\delta_q I_1}
            =\mu_0 2N^2\frac{A_z}{\delta_q} \]
            
            \\ \lasthline
        \end{longtable}
        
        %TODO ab Folie 39

        
    
   \clearpage
   \pagebreak     